\chapter{Introduction}
\thispagestyle{empty}
%\epigraph{
%Our greatest glory is not in never falling, 
%but in rising every time we fall.
%}{Confucius}%

\begin{itemize}
        \item Paragraphe qui décrit le paradigme dans lequel la recherche se conduit. 
                Donc matière noire, questions sur sa nature, rôle des 
                lentilles gravitationnelles pour révéler cette masse par la distortion
        \item Mentionner la motivation générale de notre recherche
\end{itemize}

% Possibly rework this to get a more complete picture
La recherche présentée dans ce mémoire approche le problème de cartographier la 
masse gravitationnelle de galaxies-lentilles de façon agnostique. C'est-à-dire que 
l'on ne suppose pas de modèle analytique simple pour résoudre le problème 
de reconstruction non-linéaire et mal posé. Plutôt, cette recherche exploite 
le cadre des machines à inférence récurrentielles (RIM) pour encoder des biais 
inductifs dans un réseaux de neurones qui vont rendre l'inférence de 
paramètres dans un espace à très haute dimension efficace et précise.

Avant de décrire ce travail, je vais introduire les concepts 
et les motivations nécessairent pour contextualiser ma recherche.
En premier lieu, je vais décrire le concept de lentille 
gravitationnelle à la section \ref{sec:lentilles gravitationnelles}. Ensuite, 
je vais décrire quelque concepts lié à l'extraction de profiles de masses 
provenant de large simulation magnétohydrodynamiques à la section 
\ref{sec:simulation magnetohydrodynamique}. 
Cette section sera suivit d'une introduction rapide à quelques concepts liés 
à l'apprentissage machine utiles pour ma recherche
à la section \ref{sec:apprentissage machine}. 
Je vais décrire le formalisme bayesien pour les problème inverse 
qui sous-tend ma recherche à la section \ref{sec:formalisme probleme inverse}.
Finalement, je vais décrire le 
le contexte historique et scientifique qui motive ma recherche 
à la section \ref{sec:contexte}.
%cadre plus large des motivations scientifiques 
%dans lequel cette recherche se situe à la section \ref{sec:motivations}.
\citep{Morningstar2019}

\section{Lentilles gravitationnelles}\label{sec:lentilles gravitationnelles}

L'expédition organisée par sir Arthur Eddington 
avait pour but d'observer 
l'éclipse totale du 29 mai 1919 à partir de l'île de Prìncipe 
dans le golfe de Guinée et de Sobral au nord du Brésil 
\citep{Eddington1919}. Les photographes de l'éclipses prisent , bien qu'imprécises, 
ont permis de valider la prédiction d'Einstein faite 
en 1911 que la position observée d'une étoile serait déplacée de 
$\delta \theta \approx 1.75'' \frac{R}{R_\odot}$ 
durant une eclipse \citep{Dyson1920}, soit 2 fois plus 
que ce qui est prédit par la théorie newtonienne.

% Je dois demander un cpoyright authorisation pour utiliser ces images
%\begin{figure}[H]
        %\centering
        %\begin{subfigure}[b]{0.45\linewidth}
                %\includegraphics[width=\textwidth]{figures/eddington_photo_royalsociety}
                %\caption{Crédit}
        %\end{subfigure}
        %\hfill
        %\begin{subfigure}[b]{0.45\linewidth}
                %\includegraphics[width=\linewidth]{figures/1919_results_royalsociety}
        %\end{subfigure}
%\end{figure}



Selon le principe de Fermat, les photons suivent des trajectoires qui extrémisent 
la durée de la trajectoire. Ce principe est formalisé dans le language du calcul 
des variations
\begin{equation}\label{eq:Fermat}
        \delta \int_{\lambda_A}^{\lambda_B} n(\mathbf{x}(\lambda))d\lambda = 0 
\end{equation} 
où $n$ est un indice de réfraction et $\lambda$ paramétrise la trajectoire du photon.

Pour déterminer l'indice de réfraction, on utilise le formalisme de la relativité 
générale. Un élément de distance $ds^2$ est calculé à partir d'une métrique $g_{\mu \nu}$ 
\begin{equation}\label{eq:ds}
        ds^2 = g_{\mu \nu}dx^{\mu}dx^{\nu}
\end{equation} 
Un espace plat est décrit par la métrique de Minkowsky. Pour décrire le champ 
gravitationnel d'une galaxie, on fait l'approximation que le potentiel gravitationnel 
est celui d'un fluide parfait. Opérationnellement, on entend par la que ce fluide est décrit entièrement 
par sa pression et sa densité. Le potentiel gravitationnel est ainsi déterminé 
par une équation de Poisson 
\begin{equation}\label{eq:Poisson}
       \grad^{2}\Phi = 4\pi G \rho 
\end{equation} 
Dans la limite où ce potentiel est faible $\displaystyle \frac{2\Phi}{c^{2}} \ll 1$, la 
métrique est décrite par une expansion au premier ordre autour de la 
métrique de Minkowsky pour un espace plat, de sortes que
\begin{equation}\label{eq:newton}
        ds^2 = \left( 1 + \frac{2\Phi}{c^{2}} \right)c^{2}dt^{2} - \left( 1 - \frac{2\Phi}{c^{2}} \right)(dx^{2} + dy^{2} + dz^{2})
\end{equation} 

Un photon suit une géodésique de l'espace temps $ds^{2} = 0$. Ainsi, on trouve 
l'équation
\begin{equation}\label{eq:refraction index}
        c' = \frac{|d\vec{x}|}{dt} \approx \left( 1 + \frac{2\Phi}{c^{2}} \right)c
\end{equation} 
*Peut-être un mot de plus sur comment on obtient le vecteur tangent et mentionner 
Meneghetti*

On peut alors résoudre les équations d'Euler-Lagrange qui satisfait \eqref{Fermat} 
pour déterminer que le vecteur tangent à la trajectoire des photons est 
\begin{equation}\label{eq:vecteur tangent}
        \vec{e} = -\frac{2}{c^{2}} \grad_\perp \Phi
\end{equation} 
où $\grad_\perp$ est le gradient perpendiculaire à la trajectoire du photon.
L'angle de déflection est alors l'integral sur la trajectoire du photon:
\begin{equation}\label{eq:deflection true}
        \vec{\alpha} = \frac{2}{c^{2}}\int_{\lambda_A}^{\lambda_B} \grad_\perp \Phi d\lambda
\end{equation} 

Cette intégrale est difficile à résoudre en grande partie parce que la trajectoire est courbe. 
On utilise l'approximation de Born, c'est-à-dire qu'on approxime la trajectoire 
du photon comme une ligne droite sur l'axe-$z$ avec un paramètre d'impact $b$. Cette approximation est justfiée 
dans le contexte des lentilles gravitationnelles, puisque les angles de déflection sont généralement de 
l'ordre de l'arcsecondes ou plus petit. Ainsi, en assumant que le potentiel est celui d'une masse M
ponctuelle avec $\displaystyle \Phi = -\frac{GM}{r}$,
on obtient

\begin{equation}\label{eq:deflection approx}
        \vec{\alpha}(\vec{b}) = \frac{2GM}{c^{2}} \vec{b} \int_{-\infty }^{\infty } \frac{dz}{(b^{2} + z^{2})^{3/2}} 
        = \frac{4GM}{c^{2}b} \hat{b}
\end{equation} 

Le paramètre d'impact peut être écrit en terme de la distance à la lentille et de l'angle 
observé du centre de la lentille $b = \theta D_{\ell}$. Pour une lentille mince, 
la position original d'une source de photon $\vec{\beta}$ se calcul géométriquement 
par la différence entre l'angle observé $\theta$ et l'angle de déflection $\alpha$:
\begin{equation}\label{eq:equation lentille}
       \beta = \theta - \alpha 
\end{equation} 

Finalement, on peut généraliser pour des potentiels gravitationnels générés 
par un enesemble continu de masse $dm = \Sigma d^{2}\xi'$ où $\Sigma = \int \rho dz$ et 
$d^{2} \xi'$ est la taille physique de l'élément de masse $dm$ à la position $\xi'$. 
L'angle de déflection total mesuré à un point $\xi$ est alors une convolution 
sur tout le plan de la lentille (mince) puisque l'angle de déflection dépend 
linairement de la masse $M$ (superposition des angles de déflection):
\begin{equation}\label{eq:}
        \vec{\alpha}(\vec{\xi}) = \frac{4 G}{c^{2}} \int_{\mathbb{R}^{2}} \Sigma (\xi') \frac{\xi - \xi'}{|\xi - \xi'|}d^{2}\xi'
\end{equation} 
% Show a sketch a la Schneider & Bertelmann 2001.
% Cite Meneghetti & Carrol & Morningstar?
\begin{itemize}
        \item Mentionner la dégénerscence entre la morphologie de la source et la 
                lentille
\end{itemize}

\section{Simulation magnétohydrodynamiques}\label{sec:simulation magnetohydrodynamique}
Les détails de ces simulations tombent en dehors du cadre de cette thèse, 
toutefois on peut noter que plusieurs simulations de haute qualité sont maintenant 
capable de reproduire plusieurs observables de l'Univers d'aujourd'hui ($z=0$). 
Ainsi, on peut utiliser ces simulations pour créer des profils de masses 
autrement inaccessible par des observations qui sont basées uniquement 
sur le champ électromagnétique de l'Univers.

Introduire SPL, main equation.
\begin{itemize}
        \item Devrait être courte, mais établir le ground work pour 
                SPL
        \item Décrire généralement les éléments intéressants d'Illustris, et
                comment en général ce genre de simulations sont pertinentes 
                pour notre travail -> realistic mass models.
\end{itemize}
%\subsection{Description lissée des particules}

%\subsection{Arbres kd}

\subsection{Densité de masse projeté avec lissage adaptif}
Introduire les concepts nécessaires pour justifié notre utilisation de 
lissage adaptif pour réduire les erreurs sur les angles de déflections.


\section{Apprentissage machine}\label{sec:apprentissage machine}

\subsection{Réseaux de neuronnes convolutionnels}
Mentionner les travaux de Courville et les avancées récentes.

\subsection{Réseaux de neuronnes récurrents}
Mentionner les résultats importants concernant Turing machines. 
Mentionner les détails d'un GRU convolutifs.

%\subsection{Machine à inférence récurrentielles}


\subsection{Auto-encodeur variationnel}
Inclure ici l'appendice de l'article en plus ou moins de détail.

\subsection{Transfert de l'apprentissage}
Introduction générale.
Inclure ici l'appendice de l'article?

\section{Formalisme des problèmes inverses}\label{sec:formalisme probleme inverse}

%\subsection{Approche bayesienne}

%\subsection{Biais inductifs}

%\subsection{}


\section{Contexte scientifique}\label{sec:contexte}



