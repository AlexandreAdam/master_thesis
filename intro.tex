\chapter{Introduction}
\thispagestyle{empty}
%\epigraph{
%Our greatest glory is not in never falling, 
%but in rising every time we fall.
%}{Confucius}%

%\begin{itemize}
        %\item Paragraphe qui décrit le paradigme dans lequel la recherche se conduit. 
                %Donc matière noire, questions sur sa nature, rôle des 
                %lentilles gravitationnelles pour révéler cette masse par la distortion
        %\item Mentionner la motivation générale de notre recherche
%\end{itemize}

% Possibly rework this to get a more complete picture
La recherche présentée dans ce mémoire approche le problème de cartographier la 
masse gravitationnelle de galaxies-lentilles de façon agnostique. C'est-à-dire que 
l'on ne suppose pas de modèle analytique simple pour résoudre le problème 
de reconstruction non-linéaire et mal posé. Plutôt, cette recherche exploite 
le cadre des machines à inférence récurrentielles (RIM) pour encoder des biais 
inductifs dans un réseaux de neurones qui vont rendre l'inférence de 
paramètres dans un espace à très haute dimension efficace et précise.

%Avant de décrire ce travail, je vais introduire les concepts 
%et les motivations nécessairent pour contextualiser ma recherche.
%En premier lieu, je vais décrire le concept de lentille 
%gravitationnelle à la section \ref{sec:lentilles gravitationnelles}. Ensuite, 
%je vais décrire quelque concepts lié à l'extraction de profiles de masses 
%provenant de large simulation magnétohydrodynamiques à la section 
%\ref{sec:simulation magnetohydrodynamique}. 
%Cette section sera suivit d'une introduction rapide à quelques concepts liés 
%à l'apprentissage machine utiles pour ma recherche
%à la section \ref{sec:apprentissage machine}. 
%Je vais décrire le formalisme bayesien pour les problème inverse 
%qui sous-tend ma recherche à la section \ref{sec:formalisme probleme inverse}.
%Finalement, je vais décrire 
%le contexte historique et scientifique qui motive ma recherche 
%à la section \ref{sec:contexte}.
%cadre plus large des motivations scientifiques 
%dans lequel cette recherche se situe à la section \ref{sec:motivations}.
%\citep{Morningstar2019

\section{Lentilles gravitationnelles}\label{sec:lentilles gravitationnelles}

L'idée des lentilles gravitationnelles est attribuée a Fritz \citet{Zwicky1937} %(1937) 
qui, suivant les calculs publié par \citet{Einstein1936} l'année précédente, %\citep{Einstein1936},
est le premier à postuler correctement que l'anneau d'Einstein produit par la déflection 
de la lumière d'une source en arrière plan par le champ gravitationnel d'une galaxie 
(appellée nébuleuse extra-galactique à l'époque) 
pourrait être observé. Une idée qu'Einstein lui-même considérait improbable. 
Dans le même article, il articule précisément les idées qui nous motivent encore aujourd'hui 
(presque 100 ans plus tard) à étudier ces objets, 
c'est-à-dire que les lentilles gravitationnelles permettent
\begin{itemize}
        \item d'imager des galaxies trop lointaine pour que l'on puisse les résoudre avec 
                nos téléscopes;
        \item de mesurer directement la masse gravitationnelle de ces galaxies.
\end{itemize}



\section{Simulation magnétohydrodynamiques}\label{sec:simulation magnetohydrodynamique}
Les détails de ces simulations tombent en dehors du cadre de cette thèse, 
toutefois on peut noter que plusieurs simulations de haute qualité sont maintenant 
capable de reproduire plusieurs observables de l'Univers d'aujourd'hui ($z=0$). 
Ainsi, on peut utiliser ces simulations pour créer des profils de masses 
autrement inaccessible par des observations qui sont basées uniquement 
sur le champ électromagnétique de l'Univers.

Introduire SPL, main equation.
\begin{itemize}
        \item Devrait être courte, mais établir le ground work pour 
                SPL
        \item Décrire généralement les éléments intéressants d'Illustris, et
                comment en général ce genre de simulations sont pertinentes 
                pour notre travail -> realistic mass models.
\end{itemize}
%\subsection{Description lissée des particules}

%\subsection{Arbres kd}
 
\subsection{Densité de masse projeté avec lissage adaptif}
Introduire les concepts nécessaires pour justifié notre utilisation de 
lissage adaptif pour réduire les erreurs sur les angles de déflections.


\section{Apprentissage machine}\label{sec:apprentissage machine}

\subsection{Réseaux de neuronnes convolutionnels}
Mentionner les travaux de Courville et les avancées récentes.

\subsection{Réseaux de neuronnes récurrents}
Mentionner les résultats importants concernant Turing machines. 
Mentionner les détails d'un GRU convolutifs.

%\subsection{Machine à inférence récurrentielles}


\subsection{Auto-encodeur variationnel}
Inclure ici l'appendice de l'article en plus ou moins de détail.

\subsection{Transfert de l'apprentissage}
Introduction générale.
Inclure ici l'appendice de l'article?

\section{Formalisme des problèmes inverses}\label{sec:formalisme probleme inverse}

%\subsection{Approche bayesienne}

%\subsection{Biais inductifs}

%\subsection{}


\section{Contexte scientifique}\label{sec:contexte}



