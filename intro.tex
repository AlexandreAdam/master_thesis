\chapter{Introduction}
\thispagestyle{empty}
\begin{itemize}
        \item Paragraphe qui décrit le paradigme dans lequel la recherche se conduit. 
                Donc matière noire, questions sur sa nature, rôle des 
                lentilles gravitationnelles pour révéler cette masse par la distortion
        \item Mentionner la motivation générale de notre recherche
\end{itemize}

La recherche présentée dans ce mémoire approche le problème de cartographier la 
masse gravitationnelle de galaxies-lentilles de façon agnostique. C'est-à-dire que 
l'on ne suppose pas de modèle analytique simple pour résoudre le problème 
de reconstruction non-linéaire et mal posé. Plutôt, cette recherche exploite 
le cadre des machines à inférence récurrentielles (RIM) pour encoder des biais 
inductifs dans un réseaux de neurones qui vont rendre l'inférence de 
paramètres dans un espace à très haute dimension efficace et précise.

Avant de décrire ce travail, je vais introduire les concepts 
et les motivations nécessairent pour contextualiser ma recherche.
En premier lieu, je vais décrire le concept de lentille 
gravitationnelle à la section \ref{sec:lentilles gravitationnelles}. Ensuite, 
je vais décrire quelque concepts lié à l'extraction de profiles de masses 
provenant de large simulation magnétohydrodynamiques à la section 
\ref{sec:simulation magnetohydrodynamique}. 
Cette section sera suivit d'une introduction rapide à quelques concepts liés 
à l'apprentissage machine utiles pour ma recherche
à la section \ref{sec:apprentissage machine}. 
Je vais décrire le formalisme bayesien pour les problème inverse 
qui sous-tend ma recherche à la section \ref{sec:formalisme probleme inverse}.
Finalement, je vais décrire le cadre plus large des motivations scientifiques 
dans lequel cette recherche se situe à la section \ref{sec:motivations}.


\section{Lentilles gravitationnelles}\label{sec:lentilles gravitationnelles}
Selon le principe de Fermat, les photons suivent des trajectoires qui extrémisent 
la durée de la trajectoire. Ce principe est formalisé dans le language du calcul 
des variations
\begin{equation}\label{eq:Fermat}
        \delta \int_{\lambda_A}^{\lambda_B} n(\mathbf{x}(\lambda))d\lambda = 0 
\end{equation} 
où $n$ est un indice de réfraction et $\lambda$ paramétrise la trajectoire du photon.

Pour déterminer l'indice de réfraction, on utilise le formalisme de la relativité 
générale. Un élément de distance $ds^2$ est calculé à partir d'une métrique $g_{\mu \nu}$ 
\begin{equation}\label{eq:ds}
        ds^2 = g_{\mu \nu}dx^{\mu}dx^{\nu}
\end{equation} 
Un espace plat est décrit par la métrique de Minkowsky. Pour décrire le champ 
gravitationnel d'une galaxie, on fait l'approximation que le potentiel gravitationnel 
est celui d'un fluide parfait. Opérationnellement, on entend par la que ce fluide est décrit entièrement 
par sa pression et sa densité. Le potentiel gravitationnel est ainsi déterminé 
par une équation de Poisson 
\begin{equation}\label{eq:Poisson}
       \grad^{2}\Phi = 4\pi G \rho 
\end{equation} 
Dans la limite où ce potentiel est faible $\displaystyle \frac{2\Phi}{c^{2}} \ll 1$
%\subsection{Les angles de déflection}
%\subsection{L'équation de la lentille}
\begin{itemize}
        \item Mentionner la dégénerscence entre la morphologie de la source et la 
                lentille
\end{itemize}

\section{Simulation magnétohydrodynamiques}\label{sec:simulation magnetohydrodynamique}
\begin{itemize}
        \item Devrait être courte, mais établir le ground work pour 
                SPL
        \item Décrire généralement les éléments intéressants d'Illustris, et
                comment en général ce genre de simulations sont pertinentes 
                pour notre travail -> realistic mass models.
\end{itemize}
\subsection{Description lissée des particules}

\subsection{Arbres kd}

\subsection{Densité de masse projeté avec lissage adaptif}


\section{Apprentissage machine}\label{sec:apprentissage machine}

\subsection{Réseaux de neuronnes convolutionnels}

\subsection{Réseaux de neuronnes récurrents}

\subsection{Machine à inférence récurrentielles}

\subsection{Auto-encodeur variationnel}

\subsection{Transfert de l'apprentissage}

%\acrshort{rim}

\section{Formalisme des problèmes inverses}\label{sec:formalisme probleme inverse}

\subsection{Approche bayesienne}

\subsection{Biais inductifs}

\subsection{}


\section{Motivation scientifiques}\label{sec:motivations}



