\chapter{Introduction}\label{chap:intro}
\glsaddall
\thispagestyle{empty}


Dans la quête de comprendre la naissance et l'évolution de l'Univers, la théorie de la cosmologie moderne 
est arrivée à une conclusion surprenante: seulement 4 ingrédients sont nécessaires pour 
expliquer l'ensemble des structures de l'Univers, des échelles cosmiques jusqu'à l'échelle 
des plus petites galaxies. De ces 4 ingrédients, soit les baryons, la radiation électromagnétique, 
la matière noire et l'énergie sombre, presque rien n'est connu de la matière noire et de l'énergie sombre. 
La particule de matière noire prédite par la modèle standard de la cosmologie, $\Lambda$CDM, ne 
fait pas partie du modèle standard des particules, bien qu'elle soit maintenant un ingrédient 
essentiel à notre compréhension du fond diffus cosmologique \citep{PlanckCollaboration2018} 
et de la formation de la toile cosmique et des galaxies.

La seule propriété attendue et essentielle de cette particule est qu'elle n'interagit pas avec le champ 
électromagnétique. Les halos de matières noires peuvent alors commencer à s'effondrer par 
 instabilité gravitationnelle avant la période du 
découplage du rayonnement, soit le moment où le taux de la diffusion Compton est égal au taux d'expansion de l'Univers. 
À ce moment, les particules chargées sont en mesure de se combiner pour former les premiers atomes neutres. 
Les halos de matière noire deviennent les puits de potentiel nécessaire pour accélérer l'effondrement 
gravitationnel du gaz baryonique primordial jusqu'à la production des larges structures observées aujourd'hui \citep{Lucchin2002}.

%Les lentilles gravitationnelles 
La nature élusive de la matière noire nous force à utiliser des méthodes de plus en plus sophistiquées 
pour l'étudier. Étant donné que cette particule n'interagit pas avec la lumière, nos télescopes 
ne peuvent pas détecter directement ces particules. Plutôt, on est forcé de chercher les traces 
gravitationnelles des ces halos de matière noires. Les lentilles gravitationnelles fortes, introduites 
plus en détail au chapitre \ref{sec:lentilles gravitationnelles}, devraient nous permettre 
d'accomplir précisément cet objectif à condition qu'on soit en mesure de développer les algorithmes d'inférences 
nécessaire.

Les méthodes traditionnelles pour analyser les lentilles gravitationnelles requièrent une quantité 
significative de temps d’ordinateur (de quelques heures à quelques jours), sans compter le temps des 
experts pour faire converger les analyses MCMC requises pour obtenir les paramètres d'intérêts. 
Ce problème est significatif, considérant qu'il est projeté que  
les grands relevés du ciel comme ceux qui seront menés aux observatoires Rubin et Euclid découvriront plusieurs 
centaines de milliers de lentilles gravitationnelles.  
De plus, le Télescope géant européen (ELT), faisant usage de la technologie d'optique adaptative, 
et le télescope spatial James Webb, vont nous offrir une vue sans précédent de ces systèmes, avec un 
pouvoir de résolution qui rendra possible la recherche de halo de matière noire froide \citep[p. ex.][]{Coogan2020}, 
longtemps prédite par le modèle cosmologique standard $\Lambda$CDM. 
Or, les approximations utilisées par les méthodes 
traditionnelles pour simplifier l'optimisation 
restreignent les distributions de masses considérées pour la galaxie-lentille à des lois de puissance \citep[p. ex.][]{Nightingale2018,Etherington2022}. 
Dans le régime à haute résolution des télescopes modernes, ou encore 
pour modéliser un ensemble aussi large de lentilles gravitationnelles, ces approximations deviennent 
encombrantes et peuvent biaiser l'inférence.

Dans ce mémoire, je présente une méthodologie pour l'inférence de lentilles gravitationnelles avec 
un niveau de réalisme sans précédent. Plutôt que de supposer que la distribution de masse possède 
un profil de densité simple, décrit par quelques paramètres, ou que le vrai profil est une petite perturbation 
autour de cette solution initiale \citep{Birrer2015,Birrer2018}, j'entreprends de reconstruire une image d'un profile réaliste 
provenant de la simulation cosmologique hydrodynamique IllustrisTNG \citep{Nelson2018}. 
Une telle reconstruction libre du profil a le potentiel d'automatiser l'analyse de plusieurs 
milliers d'images de lentilles gravitationnelles à haute résolution provenant des télescope 
moderne et va ouvrir une fenêtre unique l'étude des propriétés de la matière noire. 

%C'était donc un objectif très important de la 
%recherche et du développement d'algorithmes d'inférences pour les lentilles gravitationnelles.

Depuis les premières 
tentatives de reconstructions libres de profils de densité \citep{Saha1997}, 
les méthodes introduites dans le chapitre \ref{chap:intro ml} forment le premier cadre d'analyse 
complète permettant de résoudre le problème de reconstruction libre de lentilles 
gravitationnelles, un problème inverse mal-posé et non linéaire. 
Finalement, les résultats présentés aux chapitres \ref{chap:censai} montrent que notre 
méthode est suffisamment expressive pour résoudre des système complexes, avec une précision 
suffisante pour produire des reconstructions 
statistiquement significatives étant donné des observations à haut signal sur bruit et  
à haute résolution.



\section{Description du mémoire}

L'objectif principal de ce mémoire est de développer une méthode permettant de 
modéliser la distribution de masse et 
la morphologie de la source dédiée à analyser un grand nombre ($> 10^{3}$)
de lentilles gravitationnelles dans toute leur complexité et dans un temps à l'échelle humaine. 

Le chapitre \ref{sec:lentilles gravitationnelles} est une introduction aux lentilles gravitationnelles. Ensuite, 
les chapitres \ref{sec:vae} et \ref{sec:intro rim} introduisent les méthodes d'apprentissage profond 
pour résoudre le problème d'inférence introduit dans l'introduction. 
Finalement, le chapitre \ref{chap:censai} est un rapport détaillé de l'application de ces méthodes 
appliquées à des lentilles gravitationnelles simulées 
avec des profils de densités et des images de galaxies réalistes 
et des résultats sur un ensemble test. 

\section{Déclaration de l'étudiant}
Je, Alexandre Adam, déclare que l'entièreté du travail présenté dans ce mémoire est le mien. J'ai effectué la revue de 
la littérature dans ce mémoire. Lorsque j'ai utilisé des figures provenant de sources externes, j'ai clairement 
identifié le titre de la figure avec la source associée.

Pour l'article présenté au chapitre \ref{chap:censai}, j'ai modifié un code et des méthodes originellement 
développées par Laurence Perreault-Levasseur et Yashar Hezaveh pour construire des profils de masses à partir 
de la simulation IllustrisTNG, simuler des lentilles gravitationnelles à partir de ces profils et faire l'inférence avec une 
machine à inférence récurentielle. Ma contribution à ce projet est la 
production des profils de convergence à partir de la simulation IllustrisTNG, le pré-traitement 
d'un ensemble d'entraînement pour les images de galaxies à partir du champ large COSMOS, le 
développement du code d'entraînement pour la machine à inférence récurentielle et pour les auto-encodeurs 
variationnels, le développement d'un code de recherche d'hyperparamètres et d'architectures pour ces modèles, 
la production des résultats et finalement 
le développement de la méthode de réglage fin de la machine à inférence récurentielle, 
ainsi que son interprétation bayésienne.

