\chapter{$\Lambda$CDM}\label{ref:lcdm}

% Décrire la densité critique et d'ou vien Omega here.

\begin{table}[H]
        \centering
        \caption{Paramètres de $\Lambda$CDM ajusté avec les observations du fond diffus cosmologique par le téléscope Planck \citep{PlanckCollaboration2018}}
        \label{tab:cosmos}
        \begin{tabular}{ccc}
                \hline
                Paramètre & Description & Valeur \\\hline \hline
                $\Omega_{r,0}$ & Densité de la radiation & $\sim 10^{-4}$\\
                $\Omega_{m,0}$  &Densité de la matière & 0.3158 \\
                $\Omega_{c,0}h^{2}$  & Densité de la matière noire & 0.12011\\
                $\Omega_{b,0}h^{2}$ & Densité de la matière baryonique & 0.022383\\
                $\Omega_{\Lambda,0}$ &Densité de l'énergie sombre & 0.6842\\
                $\Omega_0$ & Densité totale & $\equiv 1$\\
                $h$ & Constante de Hubble $h \equiv \dfrac{H_0}{100\, \mathrm{km}\,\mathrm{s}^{-1}\, \mathrm{Mpc^{-1}}}$ & $0.6732$ \\
               \hline 
        \end{tabular}
\end{table}

