\chapter{Introduction}
\thispagestyle{empty}
\begin{itemize}
        \item Paragraphe qui décrit le paradigme dans lequel la recherche se conduit. 
                Donc matière noire, questions sur sa nature, rôle des 
                lentilles gravitationnelles pour révéler cette masse par la distortion
        \item Mentionner la motivation générale de notre recherche
        
\end{itemize}

\section{Lentilles gravitationnelles}
\subsection{Les angles de déflection}
\subsection{L'équation de la lentille}
\begin{itemize}
        \item Mentionner la dégénerscence entre la morphologie de la source et la 
                lentille
\end{itemize}

\section{Simulation magnétohydrodynamiques}
\begin{itemize}
        \item Devrait être courte, mais établir le ground work pour 
                SPL
        \item Décrire généralement les éléments intéressants d'Illustris, et
                comment en général ce genre de simulations sont pertinentes 
                pour notre travail -> realistic mass models.
\end{itemize}
\subsection{Description lissée des particules}

\subsection{Arbres kd}

\subsection{Densité de masse projeté avec lissage adaptif}


\section{Apprentissage profond}

\subsection{Réseaux de neuronnes convolutionnels}

\subsection{Réseaux de neuronnes récurrents}

\subsection{Machine à inférence récurrentielles}

\subsection{Auto-encodeur variationnel}

\subsection{Transfert de l'apprentissage}

%\acrshort{rim}

\section{Formalisme des problèmes inverses}

\subsection{Approche bayesienne}

\subsection{Biais inductifs}

\subsection{}


\section{Motivation pour cette thèse}



